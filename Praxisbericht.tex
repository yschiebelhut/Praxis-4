% ------------------------------------------------------------
% LaTeX Template für die DHBW zum Schnellstart!
% Original: https://github.wdf.sap.corp/vtgermany/LaTeX-Template-DHBW
% ------------------------------------------------------------
% ---- Präambel mit Angaben zum Dokument
\input{Inhalt/00_Latex/praeambel}

% ---- Elektronische Version oder Gedruckte Version?
% ---- Unterschied: Die elektronische Version enthält keinen Platzhalter für die Unterschrift
\usepackage{ifthen}
\newboolean{e-Abgabe}
\setboolean{e-Abgabe}{true}    % false=gedruckte Fassung

% ---- Persönlichen Daten:
\newcommand{\titel}{Redesigning and Implementing the Bootstrap of Large Scale Kubernetes Enterprise Infrastructure through Automated Self Contained CLI}
\newcommand{\titelheader}{Bootstrapping Kubernetes Infrastructure} % TOOD: is this even needed?
\newcommand{\arbeit}{Project 2b (T3\_2000)}
\newcommand{\studiengang}{Computer Science (Informatik)}
\newcommand{\studienjahr}{2020}
\newcommand{\autor}{Yannik Schiebelhut}
\newcommand{\autorReverse}{Schiebelhut, Yannik}
\newcommand{\verfassungsort}{Karlsruhe}
\newcommand{\matrikelnr}{3354235}
\newcommand{\kurs}{TINF20B1}
\newcommand{\bearbeitungsmonat}{September 2022}
\newcommand{\abgabe}{September 19, 2022}
\newcommand{\bearbeitungszeitraum}{July 4, 2022 - September 19, 2022}
\newcommand{\firmaName}{SAP SE}
\newcommand{\firmaStrasse}{Dietmar-Hopp-Allee 16}
\newcommand{\firmaPlz}{69190 Walldorf, Deutschland}
\newcommand{\betreuerFirma}{Samed Güner} % TODO: doch Jan einsetzen? TODO: Mit Umlaut?
\newcommand{\betreuerDhbw}{Prof. Dr. Johannes Freudenmann}

\input{Inhalt/00_Latex/kopfundFusszeile}

% ---- Hilfreiches
\newcommand{\zB}{z.\,B. }   % "z.B." mit kleinem Leeraum dazwischen (ohne wäre nicht korrekt)
\newcommand{\dash}{d.\,h. }
\newcommand{\Dash}{D.\,h. }

\newcommand{\code}[1]{\texttt{#1}} % Ist einfacher zu schreiben als ständig \texttt und erlaubt
                                   % Änderungen im Nachhinein, wenn man z.B. Inline-Code anders stylen möchte.

% ---- Silbentrennung (falls LaTeX defaults falsch / nicht gewünscht sind)
\hyphenation{HANA}         % anstatt HA-NA
\hyphenation{Graph-Script} % anstatt GraphS-cript

% ---- Beginn des Dokuments
\begin{document}
\setlength{\parindent}{0pt}              % Keine Paragraphen Einrückung.
                                         % Dafür haben wir den Abstand zwischen den Paragraphen.
\setcounter{secnumdepth}{2}              % Nummerierungstiefe fürs Inhaltsverzeichnis
\setcounter{tocdepth}{1}                 % Tiefe des Inhaltsverzeichnisses. Ggf. so anpassen,
                                         % dass das Verzeichnis auf eine Seite passt.
\sffamily                                % Serifenlose Schrift verwenden.

% ---- Vorspann
% ------ Titelseite
\singlespacing
\begin{otherlanguage}{ngerman}
    \thispagestyle{empty}
\begin{titlepage}
\enlargethispage{4cm}

\begin{figure}           % Logo vom Ausbildungsbetrieb und der DHBW
	% \vspace*{-5mm} % Sollte dein Titel zu lang werden, kannst du mit diesem "Hack" 
	%                  den Inhalt der Seite nach oben schieben.
	\begin{minipage}{0.49\textwidth}
		\flushleft
		\includegraphics[height=2.5cm]{Bilder/Logos/Logo_SAP.pdf} 
	\end{minipage}
	\hfill
	\begin{minipage}{0.49\textwidth}
		\flushright
		\includegraphics[height=2.5cm]{Bilder/Logos/Logo_DHBW.pdf} 
	\end{minipage}
\end{figure} 
\vspace*{0.1cm}

\begin{center}
	\huge{\textbf{\titel}}\\[1.5cm]
	\Large{\textbf{\arbeit}}\\[0.5cm]
	\normalsize{in the context of the examination for the\\[1ex] \textbf{Bachelor of Science (B.Sc.)}}\\[0.5cm]
	\Large{in \studiengang}\\[1ex]
	\normalsize{at the Cooperative State University Baden-Württemberg Karlsruhe}\\[1cm]
	\normalsize{by}\\[1ex] \Large{\textbf{\autor}} \\[1cm]
\end{center}

\begin{center}
	\vfill
	\begin{tabular}{ll}
		Submission Date:                               & \abgabe \\[0.2cm]
		Processing period:                             & \bearbeitungszeitraum \\[0.2cm]
		Matriculation number, Course:                  & \matrikelnr , \kurs \\[0.2cm]
		Training institution:                          & \firmaName \\
		                                               & \firmaStrasse \\
		                                               & \firmaPlz \\[0.2cm]
		Supervisor of the training institution:        & \betreuerFirma \\[0.2cm]
		Appraiser of the Cooperative State University: & \betreuerDhbw \\[2cm]
	\end{tabular} 
\end{center}
\end{titlepage}
  % Titelseite
\end{otherlanguage}
\newcounter{savepage}
\pagenumbering{Roman}                    % Römische Seitenzahlen
\onehalfspacing

\begin{otherlanguage}{ngerman}
    % ------ Erklärung, Sperrvermerk, Abstact
    \include{Inhalt/01_Standard/erklaerung}
    % \include{Inhalt/01_Standard/sperrvermerk}
    \renewcommand{\abstractname}{Abstract} % Veränderter Name für das Abstract
\begin{abstract}
\begin{addmargin}[1.5cm]{1.5cm}        % Erhöhte Ränder, für Abstract Look
\thispagestyle{plain}                  % Seitenzahl auf der Abstract Seite

\begin{center}
\small\textit{- English -}             % Angabe der Sprache für das Abstract
\end{center}

\vspace{0.25cm}

% LTeX: language=en-US

Large scale cloud service offerings are an important part of modern enterprise infrastructure.
The deployment of these is traditionally done by hand, requiring days of work and therefore also being costly.
The supervising department works on automizing this deployment process and related maintenance tasks with the aim to provide the interaction functionality as a self-contained \ac{cli}.

\vspace{0.25cm}

Still, the tools that are used to perform the automated deployment are not able to run without some perquisites being fulfilled.
In this project, the existing bootstrapping procedure is analyzed, redesigned and implemented to be integrated into the developed \ac{cli}.
Different approaches for interacting with involved services are introduced and compared.
Then, a prototype of the bootstrap is implemented in Go.


\end{addmargin}
\end{abstract}
    \renewcommand{\abstractname}{Abstract} % Veränderter Name für das Abstract
\begin{abstract}
\begin{addmargin}[1.5cm]{1.5cm}        % Erhöhte Ränder, für Abstract Look
\thispagestyle{plain}                  % Seitenzahl auf der Abstract Seite

\begin{center}
\small\textit{- Deutsch -}             % Angabe der Sprache für das Abstract
\end{center}

\vspace{0.25cm}

% LTeX: language=de-DE

Umfassende Cloud-Service-Angebote sind ein wichtiger Bestandteil moderner Unternehmensinfrastruktur.
Die Bereitstellung solcher Angebote erfolgt traditionell manuell, was sehr zeitaufwändig und daher kostspielig ist.
Die Abteilung, die diesen Bericht betreut, arbeitet daran, diesen Bereitstellungsprozess und die damit verbundenen Wartungsaufgaben zu automatisieren, mit dem Ziel, Funktionalität zur Interaktion mit Infrastruktur als eigenständige \ac{cli} bereitzustellen.

\vspace{0.25cm}

Die Programme, die für die automatisierte Bereitstellung verwendet werden, können nicht ausgeführt werden, ohne dass einige Voraussetzungen erfüllt sind.
Daher muss die Zielumgebung in einem Bootstrap vorbereitet werden.
In diesem Projekt wird das bestehende Bootstrap-Verfahren analysiert, neu gestaltet und implementiert, um in die entwickelte \ac{cli} integriert zu werden.
Es werden verschiedene Ansätze zur Interaktion mit den beteiligten Diensten vorgestellt und verglichen.
Anschließend wird ein Prototyp des Bootstraps in Go implementiert.

\end{addmargin}
\end{abstract}
\end{otherlanguage}

% ------ Inhaltsverzeichnis
\singlespacing
\tableofcontents

% ------ Verzeichnisse
\renewcommand*{\chapterpagestyle}{plain}
\pagestyle{plain}
% \include{Inhalt/03_Verzeichnisse/formelgroessen}
\chapter*{List of Abbreviations}
\addcontentsline{toc}{chapter}{List of Abbreviations} % Hinzufügen zum Inhaltsverzeichnis 

\begin{acronym}[BPMN] % längstes Kürzel wird verw. für den Abstand zw. Kürzel u. Text

	% Alphabetisch selbst sortieren - nicht verwendete Kürzel rausnehmen!
	\acro{acl}[ACL]{Access Control List}
	% \acro{AIR}{Adobe Integrated Runtime}
	% \acro{AJAX}{Asynchronous Javascript and XML}
	% \acro{ANSI}{American National Standards Institute}
	\acro{api}[API]{Application Programming Interface}
	% \acro{AR}{Augmented Reality}
	\acro{aws}[AWS]{Amazon Web Services}
	% \acro{BAPI}{Business Application Programming Interface}
	% \acro{BIOS}{Basic Input Output System}
	\acro{bpmn}[BPMN]{Business Process Model and Notation}
	\acro{ccwf}[ccWF]{Cross-Company Workflow}
	% \acro{CDMA}{Code Division Multiple Access}
	\acro{cli}[CLI]{command-line interface}
	\acro{csv}[CSV]{Comma-Separated Values}
	\acro{erp}[ERP]{Enterprise Resource Planning}
	\acro{hcl}[HCL]{HashiCorp Configuration Language}
	\acro{hs2}[HS2]{High Speed 2}
	% \acro{HTTP}{Hypertext Transfer Protocol}
	% \acro{HTTPS}{Hypertext Transfer Protocol Secure}
	\acro{iaas}[IaaS]{Infrastructure as a Service}
	\acro{iac}[IaC]{Infrastructure as Code}
	% \acro{IP}{Internet Protocol}
	% \acro{ISBN}{Internationale Standardbuchnummer}
	% \acrodefplural{ISBN}[ISBNs]{Internationale Standardbuchnummern}
	\acro{json}[JSON]{JavaScript Object Notation}
	\acro{kpi}[KPI]{Key Performance Indicator}
	% \acro{OData}{Open Data Protocol}
	% \acro{SDK}{Software Development Kit}
	% \acro{SEO}{Search Engine Optimization}
	\acro{sql}[SQL]{Structured Query Language}
	% \acro{SSH}{Secure Shell}
	% \acro{UEFI}{Unified Extensible Firmware Interface}
	% \acro{URI}{Uniform Resource Identifier}
	% \acro{USB}{Universal Serial Bus}
	\acro{vcs}[VCS]{Version Control System}
	% \acro{VLAN}{Virtual Local Area Network}
	% \acro{WYSISWG}{What You See Is What You Get}
	\acro{xes}[XES]{eXtensible Event Stream}
	% \acro{XSL}{Extensible Stylesheet Language}

\end{acronym}
\listoffigures                          % Erzeugen des Abbildungsverzeichnisses 
% \listoftables                           % Erzeugen des Tabellenverzeichnisses
\renewcommand{\lstlistlistingname}{List of Code Listings}
\lstlistoflistings                      % Erzeugen des Listenverzeichnisses
\setcounter{savepage}{\value{page}}


% ---- Inhalt der Arbeit
\cleardoublepage
\pagenumbering{arabic}                  % Arabische Seitenzahlen für den Hauptteil
\setlength{\parskip}{0.5\baselineskip}  % Abstand zwischen Absätzen
\rmfamily
\renewcommand*{\chapterpagestyle}{scrheadings}
\pagestyle{scrheadings}
\onehalfspacing
% \include{content goes here}
\chapter{Introduction}
\section{Motivation}
Modern enterprise infrastructure for software development often makes use of cloud service offerings to dynamically adapt infrastructure to the fast-paced world of agile development.
The setup process of some cloud service offerings at SAP is traditionally done by hand and can easily require multiple days of work to complete.
Obviously, in consequence of the required man-hours, this manual deployment is rather costly and time-consuming.
Also, humans are prune to error.
In a manual task that big, it is very likely for unforeseeable errors to occur.
Finding these mistakes can eventually take additional time.

As a conclusion, it is desirable to automate as much of a cloud service's deployment process as possible.
There are already various tools available that can be leveraged to perform an automated deployment of a declared infrastructure.
But even these automatization tools require some perquisites to run, like a technical access user with permissions to access the necessary resources and access keys to utilize this user, which still have to be created manually or with the help of bare-bones scripts.
The process of fulfilling these requirements and preparing the involved services for automated deployment is referred to as bootstrap.

\section{Goal}
Goal of this project is to simplify the bootstrap by integrating the necessary functionality into a \acf{cli} which is currently developed by the supervising department.
Also, the usage should be convenient and, beside providing login data for the cloud provider and credential store, minimal user input should be required to perform the bootstrap.
Furthermore, the newly developed \ac{cli} is required to be able to get executed by an automated build server.
The ways of user interaction have to be designed accordingly.

\section{Approach}
First, fundamentals of tools and services that are related to the work are introduced.
Most of these tools are fixed because of the current working set of the supervising department and because the target services, that have to be interacted with, are clearly defined.
Although, some conceptional thoughts have to be put into the appropriate choice of libraries and \acp{api} that are used as a bridge between the programming language and the actual remote services.
As an orientation for the bootstrapping procedure, an existing Python script will be leveraged.
This script is checked for possible improvements and modernization potential.
The newly developed procedure is then implemented as a general proof of concept.

After this, it is discussed how tests could be established to prospectively ensure the structural integrity of the developed solution.
Then, the steps necessary to integrate the solution with the \ac{cli} are presented.
Eventually, the developed bootstrap will be analyzed regarding economical aspects and possible weak points, and a conclusion will be drawn.
\chapter{Fundamentals}
\section{Go}
Go (also known as Golang) is an open source programming language that was started at Google in 2007 and initially launched in 2009.
The language was designed to face engineering challenges at Google with the goal to make it \enquote{easy to build simple, reliable and efficient software}. \cite{pike.2020, golang.github}
By now the compiled and statically typed language \cite{chris.2021} is widely used and the way it approaches network concurrency and software engineering has influenced other languages to a noticeable extent. \cite{pike.2020}
Through its structure go supports programming on various levels of abstraction.
For instance, one can embed Assembler or C code into a Go program or on the other hand combine groups of components into bigger, more complex components to realize abstract design patterns. \cite{Maurer2021}
Nowadays, Go is a popular choice for everything related to DevOps and therefor also for the development of command line tools. \cite{mike.2020}

\begin{itemize}
    \item toolchain (uniform formatter, included test suite, source code based generation of documentation)
    \item third party modules repo / go get
    \item extensive standard library
    \item (cite from golang.org tutorial)
\end{itemize}

% started in 2007 released in 2009
% open source, compiled, and statically typed programming language (https://www.freecodecamp.org/news/what-is-go-programming-language/)
% face engineering challenges at Google (https://go.dev/solutions/google/)
% goals: make it "easy to build simple, reliable and efficient software" (https://github.com/golang/go)
% by now widely used
% network concurrency and software engineering are influencing other languages (https://go.dev/solutions/google/)
% through its structure go supports programming on various layers of abstraction (cite from book here)
%     supports embedded assembler and c
%     suitable for realizing abstract design patterns

\section{Cobra}
Cobra is an open source library for Go.
Its aim is to provide developers with an easy way to create modern \ac{cli} applications.
The cobra library is being used by noticeable projects like the \ac{cli}s for Kubernetes or for GitHub.
The idea behind Cobra's intended command schema is that commands of a well constructed \ac{cli} should read like sentences.
This way, new users that are familiar with \acp{cli} in general quickly feel native because interacting with the \ac{cli} feels more natural.
In this approach, a command represents a certain action that the \ac{cli} can perform.
This action than take arguments and flags to further specify on which objects and in which way the command should take action.
With Cobra, one can also easily create nested subcommands.
This means that a before mentioned command can also be divided into multiple sub-actions to enable detailed handling of complex actions.
Further, benefits of Cobra are, among others, the automated generation of autocomplete for the most common shells as well as the ability to automatically create man pages. \cite{cobra.github, cobra.dev}


% \begin{itemize}
% 	\item open source library
%     \item modern cli applications
%     \item "namenhafte" projects like Kubernetes, GitHub CLI
%     \item subcommand-based -> explain
%     \item nested subcommands
%     \item can automatically generate autocomplete
%     \item helps in generating man pages \cite{cobra.github}
% \end{itemize}

\section{Terraform}

\section{Hyperscalers}
\subsection{AWS}

\section{Gardener}

\section{Jenkins}

\section{Kubernetes}
Kubernetes (often short: k8s) is an open source solution to ease up and automate management of container based services.
While doing so\todo{this sounds ugly}, it follows a declarative paradigm.
This means that the users just needs to describe the desired state -- for example through the use of configuration files or via the Kubernetes \ac{cli} -- and Kubernetes determines the steps by itself which are necessary to reach and maintain this state.
Kubernetes also enables users to dynamically scale their applications and services.
This means that the amount of resources, that are dedicated to an application, is adapted during runtime dependent, for example, on the current number of users.
Furthermore, Kubernetes can perform load balancing and redundancy between different instances of the same service.
\cite{Bloß2019, wasistk8s}

One instance of a Kubernetes system is called a cluster.
A Cluster is composed of multiple nodes (which usually are virtual machines or physical servers) which run the actual applications.
The interaction with a cluster is managed by the so-called \emph{Kubernetes Master}.
It is a central controlling unit.
The user actually never interacts with the nodes themselves directly.
\cite{k8skonzepte}

\section{Vault}

\section{Bootstrap}
\chapter{Conceptional Thoughts}
Before implementing the bootstrap, one of the major questions is what way should be chosen for programmatically interacting with the services \ac{aws} and Vault.
Both provide multiple possibilities.

\section{\ac{aws}}
When it comes to \ac{aws}, there are three obvious ways that could be chosen.

\paragraph{\acs{rest} \ac{api}}
Practically all required \ac{aws} functions can be accessed via its \ac{http} \ac{api}.
Since Go's standard library natively includes a \ac{http} client, utilizing this would be a very lightweight solution.
You would just have to instantiate a \ac{http} client object in Go.
This object then already has all the required functionality to send requests to the \ac{api}.
An \ac{api} call is made through a \ac{http} request with a specific method (GET, POST, PUT, DELETE etc.) to an \ac{api} endpoint.
This endpoint is specific to the operation you want to perform and represented by an \ac{url}.
Additional parameters and input data for the operation can be specified in a key value style via \ac{url} parameters, the request header, or the request body.
\ac{url} parameters can be generated with string replacement and then appended to the base \ac{url} for the \ac{aws} \ac{api}.
The request body is a bit more complex to construct.
It is basically a structure, that maps strings to basic data types or subordinate maps.
This has to be constructed as a structure within Go and can then be encoded into a format supported by the \ac{http} client.

Generally, using the \ac{http} \ac{api} would grant great flexibility because you construct all the requests on your own and therefore have detailed control over what happens without any additional layer of abstraction.
On the other hand, since multiple different \ac{api} calls are required, every single one of the needed calls would have to be manually constructed.
This is a lot of work, prune to errors that are hard to debug, and has a bad influence on the readability of the code in general because the \ac{api} calls would get prevalent to the actual program logic.

\paragraph{\ac{aws} \ac{cli}}
The \ac{aws} \ac{cli} provides a very easy and intuitive interface to the user for interacting with \ac{aws}.
Theoretically, it is intended to be explicitly installed on a system and to be used by a human user rather than programmatically.
Anyway, Go natively provides the functionality to execute commands on system level.
By this mean, also the \ac{aws} \ac{cli} could be used in the program.

But using the \ac{cli} would imply multiple drawbacks.
\acp{cli} often do not have a stable human interface and therefore the output returned by the \ac{cli} is subject to change.
This is no good if the program has to parse the output and behave according to the results because the program could break easily and unnoticed just by updating the \ac{cli}.
Although, in the special case of the \ac{aws} \ac{cli} the user can choose between several output formats including \ac{json} notation, so a changing interface probably would not be of a problem.
What is more of a concern is the fact that the \ac{cli} containing the bootstrap should be part of a container image packing various tools to work with clusters.
The \ac{aws} \ac{cli} is entirely written in Python.
If the \ac{aws} \ac{cli} should be used, Python would have to be installed into this container as well noticeably increasing the resulting image size.
Also, when run locally, the Go application would have to rely on an existing installation of the \ac{aws} \ac{cli} to function correctly or check for its existence and prompt the user to satisfy the dependency manually in case it is missing.
Just running the Go application executable would not be sufficient to perform the bootstrap.
Because of these reasons, embedding the \ac{aws} \ac{cli} into the newly created Go \ac{cli} coordinating the bootstrap should be seen as a solution of last resort.

\paragraph{\ac{aws} Go \acs*{sdk}}
The \ac{aws} \ac{sdk} is a library provided by Amazon itself to interface its \ac{aws} services.
It is not only available for Go but for a variety of different languages.
To make use of it, during development it can be acquired with \code{go get} and imported into the program.
In doing so, you specifically select the needed submodules minimizing the overhead.
Then, the \ac{sdk}'s functions can be normally used inside the Go program.

One notable pain point is the partly ambiguous documentation, dependent on the part and version of the \ac{sdk}.
For instance, while the methods for user management have well documented error codes in version 1 of the \ac{sdk}, telling you exactly what kind of errors you can expect, while version 2 -- which supposedly does a better job on error handling -- does not bother to take note on the possible error types, sometimes requiring in depth research to discover what you can expect.
Luckily, errors are not of a big concern for this project, as near to all kinds of occurring errors just mean an unrecoverable program state and cannot explicitly be handled by the application itself.
Furthermore, \acp{sdk} bring the inherent problem, that you completely rely on the provider in terms of update.
If \ac{aws} changed their \ac{api} while not touching the \ac{sdk}, the program would stop working with no way to fix it rather than waiting for Amazon to update the rest of their codebase.
On the other hand, since this is not a third party but an official \ac{sdk}, one could also see an advantage in it.
Staying with the example of the changed \ac{api} and assuming that the \ac{sdk} gets updated at the same time as the \ac{api}, our codebase would simply continue to work.
At the same time, you would be responsible to update a program utilizing the \acs{rest} \ac{api} entirely on your own.
As, in this case, the \ac{sdk} is released and maintained by Amazon their selves, and because of the introduced simplicity of working with an \ac{sdk} rather than the \ac{api} directly, this variant will be used in the following.

\section{Vault}
\chapter{Implementation}
\section{General Functionality}
\subsection{AWS Client Struct}
\subsection{Policy Generation from File}
\subsection{Key Rotation}
\subsection{Updating Keys in Vault}

\section{Unit Tests}

\section{Integration with the \ac{cli}}
\chapter{Future Work}
To finish off the bootstrap, work beyond the scope of this report is required.
Some fundamentals and considerations regarding these tasks shall be discussed here, although their realization will not be depicted.

\section{Establishing Tests for the Bootstrap}
When it comes to testing the implemented solution, it is important to distinguish different categories of tests.
On the one hand there are \emph{unit tests} and on the other there are \emph{integration tests}.

\paragraph{Unit tests} are about testing small components of code for functionality in different scenarios.
Do to so, the test often isolates the components from the remaining code.
Dependencies on other systems are usually swapped out to create this isolation.
This process is called \emph{mocking}.
In general, unit tests should not have side effects and should be completely independent of the rest of the application.
Because of this, unit tests are very fast, simpler in structure and therefore also easier to write.

\paragraph{Integration tests} on the other hand are way more complex than unit tests.
Like their name suggests, they test the integration of a module with other modules.
This is done when a unit test is not sufficient for testing some functions because of its isolation property.
Integration test do not try to mitigate side effects but consider them from the beginning.
Generally, integration tests are more complex to set up and slower than unit tests.
Also, integration test might often rely on external resources which failures are beyond the control of the developer.
As a result, it is usually desirable to use many unit tests and only few integration tests.

The difficulty with the implementation of the bootstrap at this state is that the interaction with \ac{aws} is deeply tied into the business logic of the bootstrap.
As a result, the bootstrap would require integration testing as a verification, although an integration test should only cover the actual interaction with \ac{aws} and not the bootstrap itself which should rather be checked with basic unit tests.
To cover the business logic of the bootstrap with unit tests, the bootstrap therefore has to be decoupled from the interaction with \ac{aws}.
The \ac{aws} part would then have to be covered with an integration test, although a much smaller one, which could focus solely on the third party service and would not get mixed with the departments own business logic.
To do so, there are different approaches that can be chosen from.

\paragraph{Client Interfaces} A common approach to write unit tests for something like the bootstrap would be to replace the clients of the \ac{aws} \ac{sdk} that communicate with the \ac{aws} backend with mocked clients.
These mocked clients could then fake the interaction with \ac{aws} and deliver reproducible results to test the actual business logic.
In Go, this is generally rather easy to achieve through the use of interfaces.
An interface in Go is just a definition of method headers.
Any type that implements the specified methods, automatically also implements the interface.
For making the actual clients interchangeable with mocked clients, interfaces would have to be specified, that define all \ac{sdk} methods the clients need.
Then, new types could be constructed, that implement those methods and return the desired values for mocking.
Now, instead of using the types of the actual \ac{sdk} clients when referring to the clients (in this case the types of the client variables in \autoref{code:aws-struct-1}), the interfaces would be used.
This enables the tests to replace the clients with the mocked clients without changing the program execution, because the same methods can be called but just on different objects.

This is an easy approach if only a few methods have to be mocked.
The problem in the context of this project is, that especially for the \ac{iam} client, many methods would have to be mocked individually, which is a lot of work, creates a lot of overhead for creating the interface, and it is difficult to cover all special cases and possible errors.

\paragraph{Function Pointers}
In some ways, this approach is rather similar to the aforementioned one.
In Go, functions can be stored in variables just like anything else.
So by extracting certain functionality into individual functions, instead of creating entire mock clients, just some functions could be swapped out for other functions delivering the mocked results.
To make use of this principle, another layer of abstraction would need to be implemented that wraps the \ac{sdk} methods into package scoped functions, and, if feasible, aggregates multiple \ac{sdk} methods into one wrapper.
These wrappers could be referred to with function pointers.
For unit testing, only these pointers would have to be adjusted to point to the mocked methods.
This can happen directly in the test file as in Go the tests are located in the same package.
The wrappers themselves could be verified for functionality with integration tests.

Although this approach reduces the complexity of the actual mocking by omitting the use of interfaces and rebuilding the entire clients, as a downside it would require some restructuring to the code because the calls to the \ac{sdk} methods would have to be replaced with the function pointers.

% \section{Integration with the \ac{cli}}
\chapter{Evaluation}
In the course of this project, a working prototype of the bootstrapping process required for automated cluster deployment via Terraform was successfully implemented in an existing Go project.
The scope was momentarily restricted to support \ac{aws} only.
It was discussed what means should be chosen for interacting with third party services.
The official \ac{aws} Go \ac{sdk} and the Vault \ac{http} \ac{api} were found to be the appropriate choices in the context of this project.
Different ways of policy management for \ac{aws} were compared and classified in terms of their suitability.
The functionality for the bootstrap was then implemented in a dedicated Go package.
The implementation of the package reached a state where the bootstrap is functional and could technically be included in the actual \ac{cli}, although automated testing still have to be established until it can be considered final.
Two different approaches for testing were discussed.
Since a restructuring of the code is likely to be necessary for the realization of these tests, the solution was not yet integrated into the \ac{cli} for the time being.

Even though, the solution is not yet included in the actual \ac{cli} for the most part the defined goal is met.
As the bootstrap package does not handle user authentication on its own, no user interaction is required for performing the bootstrap despite the initial call to the \ac{cli}.
As a consequence, the execution environment also does not matter.
Therefore, the \ac{cli} can be used both by a physical user and also by an automated build server.
The remaining step to reach the goal is adding the bootstrap as a \ac{cli} command which is rather easy once the actual functionality reached its final state.



% Goal of this project is to simplify the bootstrap of a cluster by integrating the necessary functionality into a \acf{cli} which is currently developed by the supervising department.
% Also, the usage should be convenient and, beside providing login data for the cloud provider and credential store, minimal user input should be required to perform the bootstrap.
% Furthermore, the newly developed \ac{cli} is required to be able to get executed by an automated build server.
% The ways of user interaction have to be designed accordingly.

% ---- Literaturverzeichnis
\cleardoublepage
\renewcommand*{\chapterpagestyle}{plain}
\pagestyle{plain}
\pagenumbering{Roman}                   % Römische Seitenzahlen
\setcounter{page}{\numexpr\value{savepage}+1}
\printbibliography[title=Bibliography]

% ---- Anhang
\appendix
\include{Inhalt/04_Inhalt/1100_appendix}
%\clearpage
%\pagenumbering{Roman}  % römische Seitenzahlen für Anhang

\newpage
\end{document}
