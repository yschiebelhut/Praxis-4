\renewcommand{\abstractname}{Abstract} % Veränderter Name für das Abstract
\begin{abstract}
\begin{addmargin}[1.5cm]{1.5cm}        % Erhöhte Ränder, für Abstract Look
\thispagestyle{plain}                  % Seitenzahl auf der Abstract Seite

\begin{center}
\small\textit{- Deutsch -}             % Angabe der Sprache für das Abstract
\end{center}

\vspace{0.25cm}

% LTeX: language=de-DE

Umfassende Cloud-Service-Angebote sind ein wichtiger Bestandteil moderner Unternehmensinfrastruktur.
Die Bereitstellung solcher Angebote erfolgt traditionell manuell, was sehr zeitaufwändig und daher kostspielig ist.
Die Abteilung, die diesen Bericht betreut, arbeitet daran, diesen Bereitstellungsprozess und die damit verbundenen Wartungsaufgaben zu automatisieren, mit dem Ziel, Funktionalität zur Interaktion mit Infrastruktur als eigenständige \ac{cli} bereitzustellen.

\vspace{0.25cm}

Die Programme, die für die automatisierte Bereitstellung verwendet werden, können nicht ausgeführt werden, ohne dass einige Voraussetzungen erfüllt sind.
Daher muss die Zielumgebung in einem Bootstrap vorbereitet werden.
In diesem Projekt wird das bestehende Bootstrap-Verfahren analysiert, neu gestaltet und implementiert, um in die entwickelte \ac{cli} integriert zu werden.
Es werden verschiedene Ansätze zur Interaktion mit den beteiligten Diensten vorgestellt und verglichen.
Anschließend wird ein Prototyp des Bootstraps in Go implementiert.

\end{addmargin}
\end{abstract}