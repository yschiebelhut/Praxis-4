\renewcommand{\abstractname}{Abstract} % Veränderter Name für das Abstract
\begin{abstract}
\begin{addmargin}[1.5cm]{1.5cm}        % Erhöhte Ränder, für Abstract Look
\thispagestyle{plain}                  % Seitenzahl auf der Abstract Seite

\begin{center}
\small\textit{- Deutsch -}             % Angabe der Sprache für das Abstract
\end{center}

\vspace{0.25cm}

Um unternehmensübergreifende Prozesse auf ein neues Niveau zu heben, arbeitet SAP an der Entwicklung von \emph{\acl*{ccwf} Collaboration}.
Die dort anfallenden, geteilten Prozessdaten sollen in \emph{Signavio Process Intelligence} importiert werden, um sie dort mittels Process Mining zu analysieren und in \aclp*{kpi} darzustellen.
Da bei \acl*{ccwf} die \emph{SAP Blockchain} zur Datenhaltung verwendet wird, stellt die Aufbereitung der Daten für eine Analyse in Signavio jedoch eine Herausforderung dar.

\vspace{0.25cm}

Im Rahmen dieser Projektarbeit werden zunächst diese Schwierigkeiten herausgearbeitet.
Anschließend wird sich mit der Konzeptionierung eines Datenintegrationsprozesses befasst.
Hierzu wird zum einen erörtert, wo die Transformation der Daten aus \acl*{ccwf} in das angestrebte Format durchgeführt werden sollte.
Andererseits werden auch verschiedene Transportwege beleuchtet, über die die aufbereiteten Daten in Signavio Process Intelligence bereitgestellt werden sollen.
Auf Basis des erarbeiteten Konzepts wird ein Prototyp der Datenintegration implementiert.


\end{addmargin}
\end{abstract}