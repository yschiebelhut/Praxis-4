\renewcommand{\abstractname}{Abstract} % Veränderter Name für das Abstract
\begin{abstract}
\begin{addmargin}[1.5cm]{1.5cm}        % Erhöhte Ränder, für Abstract Look
\thispagestyle{plain}                  % Seitenzahl auf der Abstract Seite

\begin{center}
\small\textit{- English -}             % Angabe der Sprache für das Abstract
\end{center}

\vspace{0.25cm}

% LTeX: language=en-US

Large scale cloud service offerings are an important part of modern enterprise infrastructure.
The deployment of these is traditionally done manually, which is very time-consuming and therefore costly.
The department, supervising this report, works on automizing this deployment process and related maintenance tasks with the aim to provide functionality to interact with infrastructure as a self-contained \ac{cli}.

\vspace{0.25cm}

The tools that are used to perform the automated deployment are not able to run without some prerequisites being fulfilled.
Therefore, the target environment needs to be prepared in a bootstrap.
In this project, the existing bootstrapping procedure is analyzed, redesigned and implemented to be integrated into the developed \ac{cli}.
Different approaches for interacting with involved services are introduced and compared.
Then, a prototype of the bootstrap is implemented in Go.


\end{addmargin}
\end{abstract}