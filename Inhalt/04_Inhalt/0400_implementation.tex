\chapter{Implementation}
As the supervising department already works on the \ac{cli} in which the bootstrap shall be included, the following implementation will take part inside an existing Go project.

Go programs are structured into packages that logically separate different aspects and functionalities.
Most of the development in this section is directly connected to interfacing with \ac{aws} and will therefore take place inside the newly created \code{aws} package.

The bootstrapping process will have to traverse the following steps in order:
\begin{enumerate}
    \item create an \ac{s3} bucket as a state storage for Terraform
    \item create Terraforms technical user
    \item create or update the access policy for the technical user
    \item attach the access policy to the technical user
    \item if access keys exist, check them for expiration
    \item if keys are expired or do not exist, perform a key rotation
    \item safe newly generated keys to Vault
\end{enumerate}
As this process is too long and complex to present it in all its details, excerpts will be used to exemplarily present the process of formation.

\section{AWS Client Struct}
A struct in Go is basically a collection of named values that forms a new type.
This is used to group data together and can be somewhat compared to objects in other programming languages.
Functions can be bounded to a struct type, forming methods that can perform operations on the contained data.

This principle will be used to build a custom \ac{aws} struct for performing the bootstrap.
The result can be seen in \autoref{code:aws-struct-1}.
On the one hand, the struct stores a general configuration and access clients to the needed \ac{aws} services (lines 2-5).
The configuration, for example, contains the information on which global region you want to operate (like \code{us-east-1} or \code{eu-central-1}).
The access clients are for \ac{s3}, \ac{iam}, and \ac{sts}.
The types of these fields are defined by their respective \ac{sdk} packages.
The following variables are strings needed multiple times across the different operations.

\lstinputlisting[
    language = Golang,
    firstline = 29,
    lastline = 41,
    caption = \ac{aws} struct (excerpt from aws.go),
    label = code:aws-struct-1
]{Quellcode/base/aws.go}

All the variables are not meant to be set by hand but through a constructor.
The constructor (\autoref{code:aws-constructor-1}) takes the desired region and cluster name as input parameters and ensures that the other values get set properly.
First, a configuration is loaded from default values (lines 4-9).
In this step, also the aforementioned credentials gets loaded into the configuration.
The constructor also packs the ability to return error.
For instance, errors could occur when loading the configuration (line 4).
If this is the case, \code{err} would be something different from \code{nil} (line 5), the error gets logged (line 6) and returned alongside a nil value for the \ac{aws} struct (line 7) because obviously successful generation failed.
After loading the configuration, the region is set and the other names are generated based on the given cluster name and region (lines 11-17).
In lines 19-21, new clients are constructed for \ac{s3}, \ac{iam}, \ac{sts} and persisted in the struct.

The \ac{aws} account ID is for instance required to generate \acp{arn}.
In this project, \acp{arn} are needed to access policies.
Therefore, the account ID needs to be obtained.
The \ac{sdk}'s \ac{sts} client has a method to get details on the identity of the calling user and returns a struct which also contains the account ID.
First, the identity struct is loaded (lines 24-28) then the ID is extracted and saved to the \ac{aws} struct.
As the \ac{arn} of a resource can be clearly calculated with the account ID, the resource type, and the resource name, the \ac{arn} for the policy can already be calculated and saved to the struct (line 30).
The constructor then returns the finished struct.

\lstinputlisting[
    language=Golang,
    firstline = 43,
    lastline = 76,
    caption = Constructor for the \ac{aws} Struct (excerpt from aws.go),
    label = code:aws-constructor-1
]{Quellcode/base/aws.go}

\section{Checking \ac{aws} State and Creating Objects}
The aim of this bootstrap is, that all necessary objects, users, and access rights exist in \ac{aws}.
Although, it is unclear, which of these might already exist.
So for each of those entities, one has to check whether they exist and create them if they do not.
As the process for each of these is quite similar, it would go beyond the constraints of this report to explain each of the steps in detail.
Instead, this process will be explained by the means of creating the \ac{s3} storage bucket for Terraform.

As briefly outlined above, the first action must be to check whether the bucket already exists as the bucket should not be overwritten if it already existed.
This is done with the \ac{sdk} method \code{HeadBucket} (lines 2-3) which receives the desired bucket name as input.
The method call returns an output and an error.
For checking bucket existence, only the error is relevant and saved to the variable \code{err}.
The output can be discarded and therefore only an underscore is written instead of a variable name.

The \ac{sdk} wraps all service errors as \emph{\ac{api} errors}.
To check, if and what error occurred, the error is interpreted as \code{smithy.APIError} (lines 6-8).
After this, it can be checked against the error types defined by the \ac{s3} \ac{sdk} package (line 9).
The relevant error type is \code{NotFound}.
If this error is on hand, the bucket does not exist and has to be created (lines 11-22).

Creating an \ac{s3} bucket via the \ac{sdk} is pretty simple but has a little trick to it.
The default \ac{aws} location to create buckets in is \emph{us-east-1}.
If and only if a bucket shall be created in a different location, one has to specify a so called \emph{LocationConstraint}.
Because of this, the configured region is checked to select the correct \ac{sdk} method call accordingly.
If creating the bucket is free of errors, the method returns nil at this point, otherwise the creation error is returned (lines 24-28).
If no error occurred in the first place, then this means a successful call of the \emph{HeadBucket} method and therefore it means that the specified bucket exists.
In this case, nothing happens and the method returns nil (line 33).

\lstinputlisting[
    language=Golang,
    firstline = 78,
    lastline = 111,
    caption = Creating the Terraform State Bucket (excerpt from aws.go),
    label = code:aws-bucket-creation
]{Quellcode/base/aws.go}

\section{Policy Generation from File}
In \autoref{sec:concept} the difference between managed and inline policies, and also why managed policies will be used, was already covered.
To create a policy in \ac{aws} you simply use an \ac{sdk} method similar to the one used in the predeceasing section for creating the \ac{s3} bucket.
But to do so, you need to pass a so-called \emph{policy document} to the method.
This is basically a \ac{json} string in which the access to the desired resources is specified.
In case of this bootstrap, the policy document will need to contain the specific names of the needed \ac{s3} buckets.
As these names are dependent on the specified cluster name, the \ac{json} string needs to be build accordingly.
To accomplish this, Go's standard library packs a functionality called \emph{templates}.
A template in Go is a string containing special symbols to mark the positions in the text that shall be replaced.
These symbols are generally indicated by double curly braces.
The braces then contain a key indicating with what property they should be replaced.
In the given case, just one string needs to be replaced at multiple places in the entire time.
Therefore, only the symbol \code{\{\{.\}\}} is needed.
The beginning of the policy file can be seen in \autoref{code:policy.tpl}.

\lstinputlisting[
    firstline = 1,
    lastline = 9,
    caption = Template File for the \ac{aws} Policy (excerpt),
    label = code:policy.tpl
]{Quellcode/base/terraform_policy.tpl}

But the let alone existence of this template file is not sufficient to make use of it.
A new method (\autoref{code:aws-policy-generation}) is written, to wrap the needed method calls of Go's template package to parse the template and build the actual policy from it.
First, the template string is loaded, that in this case resides in a separate file.
This gives the opportunity to change the template later on without recompiling the code.
This saves time and effort and also, you might not even be able to recompile because the source code is only available internally.
Files can be read in Go with the use of the \code{iotuil} package from the default library.
When calling the \code{ReadFile} method, the path to the file has to be given as a string and a byte slice is returned (lines 2-6).
Then, a new template can be parsed from the slice (lines 7-11).
During the parsing process, the template package checks for syntax errors in the template file and aborts execution if errors occur.
Finally, the policy gets \emph{executed}.
To do so, the string that should substitute the placeholders and a bytes buffer has to be passed to the templates \code{Execute} method (lines 13-17).
This buffer can then be converted to a string and is returned by the method (line 18).
The returned result is now ready to be used as a policy document.

\lstinputlisting[
    language=Golang,
    firstline = 164,
    lastline = 182,
    caption = Method for Policy Generation (excerpt from aws.go),
    label = code:aws-policy-generation
]{Quellcode/base/aws.go}


\section{Key Rotation}
Another important aspect in cloud computing, especially in the matter of long-term maintenance, is access key management.
Of course, it is very important to use secure keys and keep them secret.
But frequently renewal of keys may not be neglected.
That way, even if an attacker gains access to a key, this will not be a problem for long.
Although, this may sound simple, it has to be taken care that no legitimate user gets locked out of the systems during this process.
\ac{aws} therefore supports the parallel management of two access keys.
That way, you can delete and renew a key while, in the meantime, having undisturbed access to the system with the other key.
This process is called \emph{key rotation}.

The bootstrapping procedure shall perform a key rotation if keys already exist for the technical user and a key exceeds a maximum age of seven days.
The age of a key can be determined by its metadata.
\autoref{code:aws-key-rotation} shows how the check is performed on whether a new key has to be created or not.
The method takes a slice of access keys to check as parameter and returns a boolean that indicates if a new key has to be created in the following (line 1).
Since an \ac{aws} user can have anything between none and two access keys, there are three different cases to be distinguished.
The first and most basic case is when no access keys exist for a user.
In this case the length of the input slice is zero and a new key has to be created (lines 2-5).

If the Terraform user has exactly one access key, there are two possibilities.
Either the key is younger than seven days and nothing happens or the key exceeds the maximum age of seven days and a new key has to be created (lines 7-14).
The key is not deleted in this case because if this would be the very key that is currently being used for automated access through Jenkins, this would block the service out.
Instead, deleting the old key will happen the next time the bootstrapping procedure is executed.
This way, there is no danger of accidentally blocking the access of a service.

If two keys exist, the older one of the two has to be determined first.
Since it is known that there never ever will be more than two keys and the cases for zero keys and one key is already handled, it can safely be assumed that the first and second index of the input slice contains valid access keys.
Therefore, it is very easy to determine the oldest key with a simple if statement (lines 16-19).
After this, the procedure is very similar to the case with only one access key.
It is checked, if the key is older than seven days (line 20).
If this is the case, a new one has to be created (lines 29-30), although this time the old key gets deleted (lines 21-28).
If the key is younger than these seven days, again nothing happens (lines 33-34).

\lstinputlisting[
    language=Golang,
    firstline = 325,
    lastline = 359,
    caption = Key Rotation (excerpt from aws.go),
    label = code:aws-key-rotation
]{Quellcode/base/aws.go}

\section{Updating Keys in Vault}
Every time a new key gets created, it needs to be saved to Vault.
As already mentioned in \autoref{sec:concept}, there is already an implementation of a basic Vault client by the supervising department that just needs to be extended to add the functionality to add keys to Vault.
For this purpose, the method shown in \autoref{code:vault-update} is added to the existing Vault package.

The method binds to a config struct, storing information like the vault address and the path where the secret shall be created or where it can be accessed and satisfying a Vault client interface.
The access key to be saved is passed to the function as a parameter (line 1).
Then, the call to the Vault \ac{api} is prepared.
First, the endpoint \ac{url} is constructed from the Vault base address and the desired secret path (line 2).
Then, the request body containing the new secret is prepared.
Vault takes new secrets from the \emph{data} sections of \ac{json} bodies.
Therefore, a new map is created and the passed access key is mapped to the key \code{data}.
Go then provides a method, again from its default library, to convert an ambiguous map into a \ac{json} bytes slice (lines 4-6).
Then, a new \ac{http} client is constructed (line 8).
Also, a new \ac{http} request gets build.
The request will be of type \code{POST}, targeted to the determined \ac{url}, and utilizing the generated \ac{json} body (line 9).
The header of the request must contain a Vault token for authentication.
This token is already available in the config struct and can simply be added to the request header (line 10).
The \ac{http} client is then used to send the request (lines 11-15).

\lstinputlisting[
    language = Golang,
    firstline = 252,
    lastline = 267,
    caption = Saving Keys to Vault,
    label = code:vault-update
]{Quellcode/vault.go}
