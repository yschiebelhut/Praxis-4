\chapter{Implementation}
As the supervising department already works on the \ac{cli} in which the bootstrap shall be included, the following implementation will take part inside an existing Go project.

As the entire source code is rather extensive, only excerpts will be shown here.
The entire source code is provided in the appendix.

\section{General Functionality}
Go programs are structured into packages that logically separate different aspects and functionalities.
Most of the development in this section is directly connected to interfacing with \ac{aws} and will therefore take place inside the newly created \code{aws} package.

\subsection{AWS Client Struct}
A struct in Go is basically a collection of named values that forms a new type.
This is used to group data together and can be somewhat compared to objects in other programming languages.
Functions can be bounded to a struct type, forming methods that can perform operations on the contained data.

This principle will be used to build a custom \ac{aws} struct for performing the bootstrap.
The result can be seen in \autoref{code:aws-struct-1}.
On the one hand, the struct stores a general configuration and access clients to the needed \ac{aws} services (lines 2-5).
The configuration, for example, contains the information on which global region you want to operate (like \code{us-east-1} or \code{eu-central-1}).
The access clients are for \ac{s3}, \ac{iam}, and \ac{sts}.
The types of these fields are defined by their respective \ac{sdk} packages.
The following variables are strings needed multiple times across the different operations.

\lstinputlisting[
    language = Golang,
    firstline = 29,
    lastline = 41,
    caption = \ac{aws} struct (excerpt from aws.go),
    label = code:aws-struct-1
]{Quellcode/base/aws.go}

All the variables are not meant to be set by hand but through a constructor.
The constructor (\autoref{code:aws-constructor-1}) takes the desired region and cluster name as input parameters and ensures that the other values get set properly.
First, a configuration is loaded from default values (lines 4-9).
In this step, also the aforementioned credentials gets loaded into the configuration.
The constructor also packs the ability to return error.
For instance, errors could occur when loading the configuration (line 4).
If this is the case, \code{err} would be something different from \code{nil} (line 5), the error gets logged (line 6) and returned alongside a nil value for the \ac{aws} struct (line 7) because obviously successful generation failed.
After loading the configuration, the region is set and the other names are generated based on the given cluster name and region (lines 11-17).
In lines 19-21, new clients are constructed for \ac{s3}, \ac{iam}, \ac{sts} and persisted in the struct.

The \ac{aws} account ID is for instance required to generate \acp{arn}.
In this project, \acp{arn} are needed to access policies.
Therefore, the account ID needs to be obtained.
The \ac{sdk}'s \ac{sts} client has a method to get details on the identity of the calling user and returns a struct which also contains the account ID.
First, the identity struct is loaded (lines 24-28) then the ID is extracted and saved to the \ac{aws} struct.
As the \ac{arn} of a resource can be clearly calculated with the account ID, the resource type, and the resource name, the \ac{arn} for the policy can already be calculated and saved to the struct (line 30).
The constructor then returns the finished struct.

\lstinputlisting[
    language=Golang,
    firstline = 43,
    lastline = 76,
    caption = Constructor for the \ac{aws} Struct (excerpt from aws.go),
    label = code:aws-constructor-1
]{Quellcode/base/aws.go}

\subsection{Checking \ac{aws} State and Creating Objects}
The aim of this bootstrap is, that all necessary objects, users, and access rights exist in \ac{aws}.
Although, it is unclear, which of these might already exist.
So for each of those entities, one has to check whether they exist and create them if they do not.
As the process for each of these is quite similar, it would go beyond the constraints of this report to explain each of the steps in detail.
Instead, this process will be explained by the means of creating the \ac{s3} storage bucket for Terraform.

As briefly outlined above, the first action must be to check whether the bucket already exists as the bucket should not be overwritten if it already existed.
This is done with the \ac{sdk} method \code{HeadBucket} (lines 2-3) which receives the desired bucket name as input.
The method call returns an output and an error.
For checking bucket existence, only the error is relevant and saved to the variable \code{err}.
The output can be discarded and therefore only an underscore is written instead of a variable name.

The \ac{sdk} wraps all service errors as \emph{\ac{api} errors}.
To check, if and what error occurred, the error is interpreted as \code{smithy.APIError} (lines 6-8).
After this, it can be checked against the error types defined by the \ac{s3} \ac{sdk} package (line 9).
The relevant error type is \code{NotFound}.
If this error is on hand, the bucket does not exist and has to be created (lines 11-22).

Creating an \ac{s3} bucket via the \ac{sdk} is pretty simple but has a little trick to it.
The default \ac{aws} location to create buckets in is \emph{us-east-1}.
If and only if a bucket shall be created in a different location, one has to specify a so called \emph{LocationConstraint}.
Because of this, the configured region is checked to select the correct \ac{sdk} method call accordingly.
If creating the bucket is free of errors, the method returns nil at this point, otherwise the creation error is returned (lines 24-28).
If no error occurred in the first place, then this means a successful call of the \emph{HeadBucket} method and therefore it means that the specified bucket exists.
In this case, nothing happens and the method returns nil (line 33).

\lstinputlisting[
    language=Golang,
    firstline = 78,
    lastline = 111,
    caption = Creating the Terraform State Bucket (excerpt from aws.go),
    label = code:aws-bucket-creation
]{Quellcode/base/aws.go}

\subsection{Policy Generation from File}
\subsection{Key Rotation}
\subsection{Updating Keys in Vault}

\section{Unit Tests}

\section{Integration with the \ac{cli}}