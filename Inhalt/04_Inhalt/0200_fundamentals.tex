\chapter{Fundamentals}
\section{Go}
Go (also known as Golang) is an open source programming language that was started at Google in 2007 and initially launched in 2009.
The language was designed to face engineering challenges at Google with the goal to make it \enquote{easy to build simple, reliable and efficient software}. \cite{pike.2020, golang.github}
By now the compiled and statically typed language \cite{chris.2021} is widely used and the way it approaches network concurrency and software engineering has influenced other languages to a noticeable extent. \cite{pike.2020}
Through its structure go supports programming on various levels of abstraction.
For instance, one can embed Assembler or C code into a Go program or on the other hand combine groups of components into bigger, more complex components to realize abstract design patterns. \cite{Maurer2021}
Nowadays, Go is a popular choice for everything related to DevOps and therefor also for the development of command line tools. \cite{mike.2020}

\begin{itemize}
    \item toolchain (uniform formatter, included test suite, source code based generation of documentation)
    \item third party modules repo / go get
    \item extensive standard library
    \item (cite from golang.org tutorial)
\end{itemize}

% started in 2007 released in 2009
% open source, compiled, and statically typed programming language (https://www.freecodecamp.org/news/what-is-go-programming-language/)
% face engineering challenges at Google (https://go.dev/solutions/google/)
% goals: make it "easy to build simple, reliable and efficient software" (https://github.com/golang/go)
% by now widely used
% network concurrency and software engineering are influencing other languages (https://go.dev/solutions/google/)
% through its structure go supports programming on various layers of abstraction (cite from book here)
%     supports embedded assembler and c
%     suitable for realizing abstract design patterns

\section{Cobra}
Cobra is an open source library for Go.
Its aim is to provide developers with an easy way to create modern \ac{cli} applications.
The cobra library is being used by noticeable projects like the \ac{cli}s for Kubernetes or for GitHub.
The idea behind Cobra's intended command schema is that commands of a well constructed \ac{cli} should read like sentences.
This way, new users that are familiar with \acp{cli} in general quickly feel native because interacting with the \ac{cli} feels more natural.
In this approach, a command represents a certain action that the \ac{cli} can perform.
This action than take arguments and flags to further specify on which objects and in which way the command should take action.
With Cobra, one can also easily create nested subcommands.
This means that a before mentioned command can also be divided into multiple sub-actions to enable detailed handling of complex actions.
Further, benefits of Cobra are, among others, the automated generation of autocomplete for the most common shells as well as the ability to automatically create man pages. \cite{cobra.github, cobra.dev}


% \begin{itemize}
% 	\item open source library
%     \item modern cli applications
%     \item "namenhafte" projects like Kubernetes, GitHub CLI
%     \item subcommand-based -> explain
%     \item nested subcommands
%     \item can automatically generate autocomplete
%     \item helps in generating man pages \cite{cobra.github}
% \end{itemize}

\section{Terraform}

\section{Hyperscalers}
\subsection{AWS}

\section{Gardener}

\section{Jenkins}

\section{Kubernetes}
Kubernetes (often short: k8s) is an open source solution to ease up and automate management of container based services.
While doing so\todo{this sounds ugly}, it follows a declarative paradigm.
This means that the users just needs to describe the desired state -- for example through the use of configuration files or via the Kubernetes \ac{cli} -- and Kubernetes determines the steps by itself which are necessary to reach and maintain this state.
Kubernetes also enables users to dynamically scale their applications and services.
This means that the amount of resources, that are dedicated to an application, is adapted during runtime dependent, for example, on the current number of users.
Furthermore, Kubernetes can perform load balancing and redundancy between different instances of the same service.
\cite{Bloß2019, wasistk8s}

One instance of a Kubernetes system is called a cluster.
A Cluster is composed of multiple nodes (which usually are virtual machines or physical servers) which run the actual applications.
The interaction with a cluster is managed by the so-called \emph{Kubernetes Master}.
It is a central controlling unit.
The user actually never interacts with the nodes themselves directly.
\cite{k8skonzepte}

\section{Vault}

\section{Bootstrap}