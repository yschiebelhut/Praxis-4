\chapter{Conceptional Thoughts}
\label{sec:concept}
Before implementing the bootstrap, one of the major questions is what way should be chosen for programmatically interacting with the services \ac{aws} and Vault.
Both provide multiple possibilities.

\section{\ac{aws}}
When it comes to \ac{aws}, there are three obvious ways that could be chosen.

\paragraph{\acs{rest} \ac{api}}
Practically all required \ac{aws} functions can be accessed via its \ac{http} \ac{api}.
Since Go's standard library natively includes a \ac{http} client, utilizing this would be a very lightweight solution.
You would just have to instantiate a \ac{http} client object in Go.
This object then already has all the required functionality to send requests to the \ac{api}.
An \ac{api} call is made through a \ac{http} request with a specific method (GET, POST, PUT, DELETE etc.) to an \ac{api} endpoint.
This endpoint is specific to the operation you want to perform and represented by an \ac{url}.
Additional parameters and input data for the operation can be specified in a key value style via \ac{url} parameters, the request header, or the request body.
\ac{url} parameters can be generated with string replacement and then appended to the base \ac{url} for the \ac{aws} \ac{api}.
The request body is a bit more complex to construct.
It is basically a structure, that maps strings to basic data types or subordinate maps.
This has to be constructed as a structure within Go and can then be encoded into a format supported by the \ac{http} client.

Generally, using the \ac{http} \ac{api} would grant great flexibility because you construct all the requests on your own and therefore have detailed control over what happens without any additional layer of abstraction.
On the other hand, since multiple different \ac{api} calls are required, every single one of the needed calls would have to be manually constructed.
This is a lot of work, prune to errors that are hard to debug, and has a bad influence on the readability of the code in general because the \ac{api} calls would get prevalent to the actual program logic.

\paragraph{\ac{aws} \ac{cli}}
The \ac{aws} \ac{cli} provides a very easy and intuitive interface to the user for interacting with \ac{aws}.
Theoretically, it is intended to be explicitly installed on a system and to be used by a human user rather than programmatically.
Anyway, Go natively provides the functionality to execute commands on system level.
By this mean, also the \ac{aws} \ac{cli} could be used in the program.

But using the \ac{cli} would imply multiple drawbacks.
\acp{cli} often do not have a stable human interface and therefore the output returned by the \ac{cli} is subject to change.
This is no good if the program has to parse the output and behave according to the results because the program could break easily and unnoticed just by updating the \ac{cli}.
Although, in the special case of the \ac{aws} \ac{cli} the user can choose between several output formats including \ac{json} notation, so a changing interface probably would not be of a problem.
What is more of a concern is the fact that the \ac{cli} containing the bootstrap should be part of a container image packing various tools to work with clusters.
The \ac{aws} \ac{cli} is entirely written in Python.
If the \ac{aws} \ac{cli} should be used, Python would have to be installed into this container as well noticeably increasing the resulting image size.
Also, when run locally, the Go application would have to rely on an existing installation of the \ac{aws} \ac{cli} to function correctly or check for its existence and prompt the user to satisfy the dependency manually in case it is missing.
Just running the Go application executable would not be sufficient to perform the bootstrap.
Because of these reasons, embedding the \ac{aws} \ac{cli} into the newly created Go \ac{cli} coordinating the bootstrap should be seen as a solution of last resort.

\paragraph{\ac{aws} Go \acs*{sdk}}
The \ac{aws} \ac{sdk} is a library provided by Amazon itself to interface its \ac{aws} services.
It is not only available for Go but for a variety of different languages.
To make use of it, during development it can be acquired with \code{go get} and imported into the program.
In doing so, you specifically select the needed submodules minimizing the overhead.
Then, the \ac{sdk}'s functions can be normally used inside the Go program.

One notable pain point is the partly ambiguous documentation, dependent on the part and version of the \ac{sdk}.
For instance, while the methods for user management have well documented error codes in version 1 of the \ac{sdk}, telling you exactly what kind of errors you can expect, while version 2 -- which supposedly does a better job on error handling -- does not bother to take note on the possible error types, sometimes requiring in depth research to discover what you can expect.
Luckily, errors are not of a big concern for this project, as near to all kinds of occurring errors just mean an unrecoverable program state and cannot explicitly be handled by the application itself.
Furthermore, \acp{sdk} bring the inherent problem, that you completely rely on the provider in terms of update.
If \ac{aws} changed their \ac{api} while not touching the \ac{sdk}, the program would stop working with no way to fix it rather than waiting for Amazon to update the rest of their codebase.
On the other hand, since this is not a third party but an official \ac{sdk}, one could also see an advantage in it.
Staying with the example of the changed \ac{api} and assuming that the \ac{sdk} gets updated at the same time as the \ac{api}, our codebase would simply continue to work.
At the same time, you would be responsible to update a program utilizing the \acs{rest} \ac{api} entirely on your own.
As, in this case, the \ac{sdk} is released and maintained by Amazon their selves, and because of the introduced simplicity of working with an \ac{sdk} rather than the \ac{api} directly, this variant will be used in the following.

\section{Vault}
The aspects and options to consider for Vault are very similar to those for \ac{aws}.
Vault also has an \ac{http} api, a \ac{cli} and an own Go \ac{sdk}.
One noticeable difference here is that -- even though the \ac{sdk} is provided by the vendor itself -- the documentation for the \ac{sdk} is not good at all, especially in terms of login.
To get some login methods running, one has to dig rather deep.
Some functionality simply is not documented at all.

But the by far more important points are, that the newly created \ac{cli} by the supervising department already contains some work in progress parts that interact with Vault.
It would not make much sense to reimplement key aspects that these solutions already cover.
Furthermore, mixing different approaches to access Vault would be bad practice as it makes things unnecessarily complex and introduced multiple points of failure for key functionality.
Because of this, Vault's \ac{http} \ac{api} will be used in this project.
This can be additionally reasoned with the advantages of \ac{http} \acp{api} described in the predeceasing section with the difference, that there are not many functions of Vault that need to be accessed despite reading and writing secrets.
The programming overhead in writing these interactions manually via the \ac{http} client is pretty small.

\section{Assumptions}
The before mentioned container image already contains scripts to manage login to \ac{aws}.
These scripts take a \ac{saml} response and generate the file \code{\textasciitilde/.aws/credentials} out of it.
This file contains sets of \ac{aws} credentials consisting of an access key ID, a session token and a secret key.
These credential sets are sufficient authentication information to interface \ac{aws}.
For the following implementation, the assumption is made, that this file already exists and contains a valid key set.

\section{Policy Management}
In \ac{aws} a policy grants access to specified resources like \ac{s3} storage buckets.
During the bootstrap, a technical service user is created for Terraform which needs the rights to access the Terraform state bucket.
To grant these permissions, a policy is needed.
There are different types of policy that can be chosen from, the two most noticeable ones being \emph{managed policies} and \emph{inline user policies}.
A managed policy is an own entity that is created and can exist independently of a user or similar objects.
To grant a user the rights that the policy grants, the policy is \emph{attached} to the user.
An inline policy on the other hand is a property of the user it is attached to.
In does only apply to this single user and cannot be attached to different users, groups or roles.

The policy created in the bootstrap will only be used for the terraform user.
Therefore, theoretically it would be the proper way to go to use an inline policy so that the list of other policies does not get cluttered unnecessarily.
Although, different size limits apply to the different policy types.
The total size of all inline policies that can be attached to a user must not exceed 2048 bytes.
Because the cluster name is used multiple times in the policy, if you do the calculation, it becomes clear that the usage of inline policies would entail harsh length limits on the cluster name that would not be reasonable.
Therefore, managed policies will be used even though the management has a bigger overhead.